%%----------------------------------------------------------------------------------------
\pagestyle{plain}
%%----------------------------------------------------------------------------------------
%%       PREFAZIONE
%%----------------------------------------------------------------------------------------
\part*{Prefazione}
I \emph{pezzi} di cui un aereo è fatto sono collegati fra loro mediante chiodature, bullonature, incollaggi, saldature; quando l'aereo è in volo (e soprattutto in fase di manovra) forze di vario genere sollecitano \emph{pezzi} e \emph{collegamenti}: se questi sono stati costruiti senza un'adeguata metodologia, si può verificare una delle seguenti circostanze, sulla cui deprecabilità non vi è alcun dubbio: 
%----------------------------------------------------------------------------------------
\begin{itemize}
\item \emph{pezzi} e \emph{collegamenti} sono molto resistenti, ma molto \emph{grossi} (sovradimensionati) e l'aereo pesa troppo;
\item \emph{pezzi} e \emph{collegamenti} sono molto esili e leggeri ma si romperanno (o si deformeranno tanto da compremettere il funzionamento dell'aereo).
\end{itemize}
%----------------------------------------------------------------------------------------
Lo strutturista è chiamato a garantire che non si verifichi nessuna delle circostanze appena elencate; ciò si traduce nel fatto che egli deve essere in grado di gestire i seguenti due problemi:
%----------------------------------------------------------------------------------------
\begin{description}
\item[Verifica:] dato un elemento strutturale (\emph{pezzo} o \emph{collegamento}) di materiale e geometria noti, soggetto a carichi noti, determinare gli effetti che i carichi provocano in esso e, alla luce di questi ultimi, controllare che l'elemento (il \emph{pezzo}) resista con adeguati margini di sicurezza;
\item[Dimensionamento:] nota la forma di un elemento strutturale, prefissati alcuni parametri geometrici (quelli imposti dal suo \emph{ruolo}), scelto il materiale e noti i carichi che lo solleciteranno, individuare quali valori devono avere le sue dimensioni ancora \emph{in bianco} (cioè, che devono ancora essere determinate) affinché esso possa esercitare le proprie funzioni col minimo peso e adeguati margini di sicurezza.
\end{description}
%----------------------------------------------------------------------------------------
%\footnotesize{
%\paragraph{INDICAZIONI PER LO SVILUPPO DELLE ESERCITAZIONI A CASA\@.}~
%\\
%Il rispetto di queste indicazioni è tassativo. In presenza di difformità non prenderò in considerazione le relazioni. Ogni cosa riportata va letta con molta attenzione prima di essere sottoposta alla mia attenzione: non conviene ''usare'' un docente come correttore di bozze. 
%%----------------------------------------------------------------------------------------
%\paragraph{STESURA DEL TESTO (CON O SENZA WORD PROCESSOR)\@.} È richiesta un'esposizione strutturata piuttosto che narrativa. Pertanto descrivere sinteticamente ed in sequenza 
%%----------------------------------------------------------------------------------------
%\begin{itemize}
%\item lo scopo
%\item lo sviluppo
%\item l'applicazione
%\item le conclusioni
%\end{itemize}
%%----------------------------------------------------------------------------------------
%indicando gli strumenti (tecnici, informatici o scientifici) utilizzati per lo sviluppo e la stesura. È vietato riprodurre, anche in parte, la teoria alla base dell'esercizio: limitarsi all'indicazione bibliografica. La lunghezza, in facciate, del corpo del resoconto del lavoro a casa (escludendo quindi titolo, indice e lista dei simboli) va contenuta al massimo. 
%%----------------------------------------------------------------------------------------
%\paragraph{INDICAZIONI PARTICOLARI\@.} Il fascicolo che contiene gli esercizi deve essere curato, preciso, elegante, e pertanto
%%----------------------------------------------------------------------------------------
%\begin{itemize}
%\item i risultati numerici vanno riportati con la giusta accuratezza: porre \textbf{ESTREMA} attenzione all'aspetto delle cifre significative
%\item ogni rappresentazione grafica deve essere pertinente
%\item riportare sempre il sommario dei risultati in quadri sinottici od in opportuni grafici
%\item \textbf{FIGURE/DIAGRAMMI}\@. Figure in bianco, nero e toni di grigio (immagini e foto riprese da sorgenti bibliografiche, compresa la rete, potranno essere a colori). Inserire nel testo oppure alla fine, numerando e spaziando per bene, nel rispetto e con indicazione delle scale, con una legenda esauriente (= con tutte le indicazioni), senza sovrapporre la legenda ai grafici, usare simboli adeguatamente grandi. Il formato deve essere umano e l'assetto verticale. Ogni risultato in figura va commentato (nel testo od anche in didascalia). Il $C_d/C_D$ va misurato in \emph{Drag Count} e parte sempre da zero (lo stesso vale per la resistenza), ingrandire le polari nelle regioni di bassa resistenza
%\item Il disegno del profilo: \textbf{LE SCALE} (!), produrre una figura della larghezza utile della pagina, il tratto deve essere “corretto”
%\item evitare per quanto possibile termini in lingua diversa dall'italiano (un termine irrinunciabile di altra lingua va scritto in corsivo), evitare \emph{tout court} versioni italianizzate di termini di altre lingue
%\item nella stesura informatica lasciare un spazio bianco dopo i caratteri .\@,\@;\@?\@!\@; in stampa lasciare 3.5 cm a sinistra, 2 cm a destra
%\item eventuali formule vanno numerate
%\item non è necessario (ma può essere utile) riportare la lista dei simboli
%\item impiegare sempre una terminologia appropriata
%\item stare attenti ad evitare il costrutto “: (due punti) seguito da una figura o da una tabella”
%\item \textbf{CFD}\@. Le scale in toni di grigio. Congruità dei confronti con Xfoil: parità di $C_l$, rispetto dei limiti di validità 
%\item Scrivere sempre “numero di” Mach/Reynolds e non “Mach/Reynolds”.
%\end{itemize}
%%----------------------------------------------------------------------------------------
%\paragraph{PRESENTAZIONE\@.} Esercizi ed elaborati vanno presentati in un fascicolo non rilegato, indicando in copertina cognome, nome e matricola, insieme all’elenco di tutti gli esercizi in sviluppo o già convalidati, e riportando in seconda pagina le \textbf{INDICAZIONI PER LO SVILUPPO DELLE ESERCITAZIONI A CASA}\@. La forma è da me valutata in modo paritetico rispetto ai contenuti (e dunque leggere ogni cosa con molta attenzione prima di sottopormela). 
%%----------------------------------------------------------------------------------------
%\paragraph{CONTROLLO e CORREZIONE\@.} Io controllo e correggo solo testi~--~parziali o completi~--~purché già scritti in una forma definitiva (i.e.\ , non in bozza). Ovviamente il proponente procederà ad una preliminare autoverifica anche (e sopratutto) per gli aspetti formali\dots \\
%Interromperò il controllo di un esercizio alla prima violazione di una delle regole sopra riportate. È possibile sottopormi via mail il testo da controllare (in formato .pdf, dimensione $<500\,\,kb$).
%%----------------------------------------------------------------------------------------
%}
%%----------------------------------------------------------------------------------------
